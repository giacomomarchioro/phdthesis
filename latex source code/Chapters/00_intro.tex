\chapter*{Introduction}
The spatial-analysis of physico-chemical data of an object is a topic of research in many different fields. In this dissertation we focused on the applications on the field of Cultural Heritage where a very large variety of objects with different shapes and material are studied and hence special solutions have to be implemented in a continual challenge against their deterioration. When the sample is not homogeneous or environmental factors may affect and transform heterogeneously  the object depending on the position, a random sampling of the object can be misleading or hide position depended correlations crucial for the understanding of the materials and their deterioration. In this case associating the measurement to the point where has been collected is mandatory. This process can be done in post processing using different software techniques for registering\footnote{the term registering in this context means referencing the data over a model representing a greater part of the object} the acquisition, or during the acquisition itself, hence implementing the hardware necessary to carry on the acquisition in the area selected. The hardware solution is often preferable even though more complex and expensive to implement.  Both technique require however a model of the object, or at least a detailed representation, and this step  has to be performed often with other techniques before the acquisition of the analyses to be referenced.
Once we have coupled several instruments to the mechanical system we can obtain what in this dissertation we will call \textbf{multi-modal system}. We define a multi-modal system as a system that allows to use different analytical techniques using the same positioning system hence allowing a fine registration of the data\footnote{multi-technique system may seems more appropriate when using completely different techniques, however this term is not in use.}. However, creating a multi-modal system should not be limited to assembly together different instruments but also to fuse the data in an integrated representation.  Many times this should include also monitoring the environmental conditions, even though in this case rather than a spatial registration we perform a temporal synchronization. In fact, when measuring features at micron or nano-scale, vibrations, or slight variation of the temperature and humidity, can affect the performance, the position of the instrument and the object morphology and properties.
Eventually a data-fusion process should be performed to allow to use and exchange the data collected not only between the researchers who performed the analysis but also by the researchers who may be interested in the data-set.  To carry out this step is necessary of an understanding of the underlying data-structure and an effective documentation of the data. This last passage can be critical, because it's often very time consuming because the meta and auxiliary data relative to the samples have to be added at the measurement performed. Furthermore 

In the field of Cultural Heritage the data must be discussed and interpreted not only by scientist but also with conservator and art historian.
%To address to this step a rudimental laboratory information managment system has been developed, so that the documentation and description of the data-set can be automated.

Furthermore the different terms borrow from other disciplines.

The visualization is an important step of the scientific research, computer graphics when Mandelbrot proposed the concept of fractal geometries was working at the IBM facility and was able to visualize using computer graphic years of reaserch done on self-similar geometries.


\section*{Trend towards the same direction}
The american Libray of Congress in 2019 announced that the
Preservation Research and Testing Division (PRTD) lunch the a new infrastructure initiative call the Center for Library Analytical Scientific Samples Digital (CLASS-D) is a transformative methodology that links multiple analyses to one sample or object, and potentially multiple samples in a project to previous longitudinal research studies.

\section*{External data}
Different tools have been used to share open data and products. 
web-based hosting service

Kaggle datasets 

GrabCAD

Github

During this dissertation some special boxes with icons are used to indicate that some external content relevant to the section is hosted in one of the web-based hosting services listed above. The external content is categorized as follows:
\begin{datasetexc}
The dataset box indicate that a dataset collected during the research project is accessible online with a short description and a link to access them.
\end{datasetexc}
\begin{pythonexc}
The Python box is used for linking to external software libraries developed by the author. Libraries from third parties are cited in the bibliography.   
\end{pythonexc}
\begin{jupyterexc}
This box indicates that is available an interactive notebook for didactic purposes that uses Python for implementing the concepts and procedure explained in the section.
\end{jupyterexc}
\begin{freecadexc}
This box indicates that is available some 3D model in \texttt{.FCStd} or \texttt{.step} of the custom pieces of the instruments designed for implementing the systems used in the research project. 
\end{freecadexc}
\section*{Who could benefit from reading this dissertation}
This work was intended to link different field of knowledge and made understandable the information to a variety of professional figures ( chemists,computer scientists, engineers and conservators), providing an accessible overview of the techniques, problems and the lexicon used for implementing acquisition of both morphology and physico-chemical properties of an object. In general, if you are using a scanning system you may find useful some concept in section and if you are using a conoscopic holography sensor, this thesis may show you different applications.
Eventhough for practical reason we separate our field of knowledge, creating huge babel towers the prefix uni- of the word university do not stand for the whole but for the one.