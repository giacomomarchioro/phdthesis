\chapter{Multi-modal scanning system}

An alternative to a fully automated mechanical arm could  a portable articulated measuring arm
%https://www.hexagonmi.com/products/portable-measuring-arms/absolute-arm-7-axis

precision motion control
\section{Linear stages basics concepts}
Some basics regarding the linear stage


Screw Actuators:
Lead screw actuators: - A lead screw actuator with a threaded nut that moves with a screw. It provides both simple construction and lowest cost.

Ball screw actuators: - A lead screw actuator that uses ball bearings; more expensive but less friction compared with lead screw actuators.

Planetary roller screw actuators: - A lead screw actuator that uses threaded rollers surrounding the main threaded shaft; the most expensive option but also the most durable.

Belt Actuators:
Actuators based on belt drives; often used where speed is important but with limited accuracy.

Rod vs. Rodless Actuators:
Rod type: - The thrust element or rod moves out of the end of the actuator as motion takes place; produces more force and highly tolerant of dirty environments but require a bearing structure to carry the load.

Rodless type: - The actuator housing completely surrounds the screw which provides a load bearing and guidance structure. Rodless actuators are difficult to seal for dirty or wet environments.

Integrated Actuators:
Integrated Actuators integrate a rod style actuator into a motor housing to eliminate the need for a coupling. Integrated actuators provide the lowest size and weight and provide easier maintenance but are typically higher cost.

Screw Driven Positioning Stages:
Screw driven positioning stages are used in applications where accuracy  repeatability are more critical than axial thrust forces. The base, carriage, and all sub-assembly components are precision machined which contribute to the accuracy and repeatability of the stage. These positioning tables use either an acme or ball screw as the drive mechanism. Ball  rod, cross roller, round rail, or square rail linear bearings are used to carry the user load. These linear bearing designs allow the user load to be positioned very accurately & repeatedly.

Belt Driven Positioning Stages:
Belt driven positioning tables are used in high speed (and/or long travel) positioning applications where a screw driven stage is not practical, usually in stroke lengths over 6ft. The belt & pulley drive mechanism, along with either round rail or square rail linear bearings, provide a modestly repeatable high speed positioning table. Unique to belt driven positioning stages is their ability to provide the same speed capability independent of travel length.
\subsection{Scanning types}


\subsection{Scanning patterns}
There are different scanning patters that can be performed \ref{}. It's very important to define a common definition of the patterns, this is essential when the data is stored as a flatten matrix, and the user has to reshape the data to its original dimension. The classical \textbf{raster scanning} is usually not convenient when the time for acquiring a single line is great
\section{Integration with other sensor}

\section{The conoscopic holography sensor}