\chapter*{Preface}
I developed most of the topics written in this thesis during the European Project Scan4Reco (Multimodal Scanning of Cultural Heritage Assets for their multilayered digitization and preventive conservation via spatiotemporal 4D Reconstruction and 3D Printing) coordinated by Dr. Dimitrios Tzovaras. The project, ended in 2018, was structured in mainly three phases: 1) create and analyze reference material coupons artificially aged 2) investigate the degradation mechanism and possible aeging models and 3) build a mechanical positioning system that allowed to investigate an artwork with different analytical techniques (e.g. Raman, FT-IR etc. etc.) referenced over a 3D model of the object. The knowledge gained during the first two phases have been eventually used by a ''decision support system'' and a simulation environment for helping the restorer and the conservation scientist to preserve the artwork. \par In this thesis I have tried to implement solutions that could address some problems faced during this project, eventually I've try to scale the architecture of the project in order to make it more accessible. In particular most of the effort have been spend in developing solutions for the creation, the characterization and the exchange of data-set of materials. %Special emphasis has been placed on
For exchanging the information however also the jargon must be defined 

A larger version of the mechanical system, that integrate the microprofilometer module developed in our lab, was developed in the Ormylia Art Diagnosis Centre by the team lead by Dr.Ing.Georgios Karagiannis while a commercial solution is under development by AVASHA AG. CERTH (The Centre for Research l\& Technology, Hellas) implemented aging simulation and ROS system and software for the multi-modal acquisition driving the system. 
